%% \usepackage[utf8]{inputenc}
%% \usepackage[T1]{fontenc}
%% \usepackage{geometry}
%% \usepackage{xcolor}
%% \usepackage{amsmath, amssymb}
%% \usepackage{tikz}
%% \usepackage{listings} % For code
%% \usepackage{tcolorbox} % For definitions
%% \usepackage{marginnote} % For better sidenotes

%% % --- LAYOUT CONFIGURATION ---
%% % 1. This handles the "Newlines" issue.
%% % It separates paragraphs by space, not indentation.
%% % You just need to leave ONE empty line in the .tex file to separate thoughts.
%% \usepackage[parskip]{parskip} 

%% % 2. Margins setup to allow space for sidenotes
%% \geometry{
%%     left=2cm,
%%     right=5cm, % Extra space on right for notes
%%     top=2cm,
%%     bottom=2cm,
%%     marginparwidth=4cm,
%%     marginparsep=0.5cm
%% }

%% % --- OCAML CODE STYLING ---
%% \definecolor{codegreen}{rgb}{0,0.6,0}
%% \definecolor{codegray}{rgb}{0.5,0.5,0.5}
%% \definecolor{codepurple}{rgb}{0.58,0,0.82}
%% \definecolor{backcolour}{rgb}{0.95,0.95,0.92}

%% \lstdefinelanguage{OCaml}{
%%   keywords={let, rec, match, with, type, of, if, then, else, begin, end, val, fun, function, module, struct, sig},
%%   keywordstyle=\color{blue}\bfseries,
%%   ndkeywords={int, string, float, bool, list},
%%   ndkeywordstyle=\color{codepurple}\bfseries,
%%   identifierstyle=\color{black},
%%   sensitive=true,
%%   comment=[n]{(*}{*)},
%%   commentstyle=\color{codegreen}\ttfamily,
%%   stringstyle=\color{red}\ttfamily,
%%   morestring=[b]"
%% }

%% \lstset{
%%     backgroundcolor=\color{backcolour},   
%%     basicstyle=\ttfamily\small,
%%     breakatwhitespace=false,         
%%     breaklines=true,                 
%%     captionpos=b,                    
%%     keepspaces=true,                 
%%     numbers=left,                    
%%     numbersep=5pt,                  
%%     showspaces=false,                
%%     showstringspaces=false,
%%     showtabs=false,                  
%%     tabsize=2,
%%     frame=lines
%% }

% Shortcut for inline code (e.g. "use the `let` keyword")
\newcommand{\code}[1]{\texttt{\color{blue}#1}}

% Environment for OCaml blocks
\lstnewenvironment{ocaml}
{\lstset{language=OCaml}}
{}

% --- DEFINITION & CONCEPT BOXES ---
% Replaces your \lemma command with a nice box
\newtcolorbox{concept}[1]{
  colback=blue!5!white,
  colframe=blue!75!black,
  title=\textbf{DEFINITION: #1},
  sharp corners=south,
  boxrule=0.5mm
}

\newtcolorbox{warning}{
  colback=red!5!white,
  colframe=red!75!black,
  title=\textbf{WATCH OUT!},
  boxrule=0.5mm
}

% --- SIDENOTES ---
% This fixes your TikZ issue. This flows with the text.
\newcommand{\sidenote}[1]{%
  \marginnote{\footnotesize \textit{\color{blue!70!black} $\rightarrow$ #1}}%
}

% --- MATH & UTILS ---
\newcommand{\vct}[1]{\overrightarrow{#1}}
\newcommand{\prf}[1]{\noindent\underline{\textsc{Proof:}} #1}
\newcommand{\mtx}[3]{#1_{#2,#3}}

% proof stuff, fix later, error is "LaTeX Error: No counter '?' defined"
%% \newtheorem{cor}{Corollary}
%% \numberwithin{cor}{section}

%% \newtheorem{prop}{Proposition}
%% \numberwithin{prop}{section}

%% \newtheorem{lem}{Lemma}
%% \numberwithin{lem}{section}

%% \newtheorem{thm}{Theorem}
%% \numberwithin{thm}{section}

% --- CUSTOM SECTIONS ---
% This allows you to type \topic{Recursion}
%% \newcommand{\topic}[1]{\section*{#1}\addcontentsline{toc}{section}{#1}}
%% \newcommand{\subtopic}[1]{\subsection*{#1}}

% --- THE "STRICT NEWLINE" ENV ---
% Use this environment if you want text to break exactly where you hit enter
% without needing double spaces.
\newenvironment{stricttext}
{\par\begingroup\obeylines}
{\endgroup\par}

%% %-*-latex-*-
%% %misc
\newcommand{\dated}[1]{\textbf{\underline{\small{$-->$ #1}}} \normalsize\\}

%% %math cmds
%% \newcommand{\vct}[1]{\overrightarrow{#1}}
%% \newcommand{\prf}[1]{\textlg{\underline{\textsc{Proof:}} } #1}
%% %this is just for declaring a matrix M of size i-by-j, or specifying a point in the matrix
%% \newcommand{\mtx}[3]{$#1_{#2,#3}$}

%% %font adjuster cmds
\newcommand{\texttn}[1]{\tiny {#1}\normalsize}
\newcommand{\textsm}[1]{\small {#1}\normalsize}
\newcommand{\textlg}[1]{\large {#1}\normalsize}
\newcommand{\texthg}[1]{\huge {#1}\normalsize}
\newcommand{\EMPHASIZE}[1]{{\huge \textbf{#1}}}
\newcommand\redtext[1]{\textcolor{red}{#1}}
\newcommand\textred[1]{\textcolor{red}{#1}}
\newcommand\bluetext[1]{\textcolor{blue}{#1}}
\newcommand\textblue[1]{\textcolor{blue}{#1}}
\newcommand\whitetext[1]{\textcolor{white}{#1}}
\newcommand\textwhite[1]{\textcolor{white}{#1}}


%% %orientation and definition cmds
\newcommand{\lemma}[1]{\underline{\texttt{DEFINITION:}} \textit{#1}\\}
\newcommand{\newsection}[1]{\newpage\textbf{\texthg{#1}}\\}
\newcommand{\topic}[1]{\textbf\underline{\textlg{\texttt{TOPIC:} #1}}\\\\}
\newcommand{\subtext}[1]{\textit{\texttn{#1}}}
\newcommand{\centerthis}[1]{\begin{center}#1\end{center}}
\newcommand{\exercise}[1]{\textbf{Exercise: \underline{#1 }\\\\ }\textsc{Answer:}\\\\}

%% %verbatim env           
\lstnewenvironment{console}[1][]
{\lstset{
    basicstyle=\ttfamily,
    backgroundcolor=\color{white},
    frame=single,
    framesep=3pt,
    rulecolor=\color{black},
    showspaces=false,    
    showtabs=false,      
    keepspaces=true,     
    #1
}}
{}

%drawing commands (boxes, basic shapes, formatting etc)

\newcommand\xsidebox[5][1=0cm, 2=0cm, 3=]{
  % #1 x-shift          -- optional (default: 0cm)
  % #2 y-shift          -- optional (default: 0cm)
  % #3 node attributes  -- optional (default: empty)
  % #4 node name
  % #5 node text
  \renewcommand*{\marginfont}{\normalsize}
  \marginpar{\centering%
      %\vspace{#1}\hspace{#2} 
      \begin{tikzpicture}[overlay,remember picture, baseline=(#4.base)]
        \node[rectangle,
          draw=red,
          text width=4cm, minimum width=2cm, line width=0.1cm, inner sep=0.2cm, name=#4, xshift=#1, yshift=#2, #3] (#4) 
             {
               \begin{minipage}[t]{4.0cm}
                 {#5}
               \end{minipage}
             };
        ([xshift=0.5]#4.center)[draw]
      \end{tikzpicture}
    \renewcommand*{\marginfont}{\tiny}
}}

\newcommand*{\DrawArrow}[3][red]{%
    % #1 = draw options
    % #2 = left node
    % #3 = right node
  \begin{tikzpicture}[overlay,remember picture]
    \draw [-latex, #1, line width=0.1cm] (#2) to (#3);
  \end{tikzpicture}%
}
\newcommand*{\DrawArrowH}[3][red]{%
    % #1 = draw options
    % #2 = left node
    % #3 = right node
  \begin{tikzpicture}[overlay,remember picture]
    \draw [-latex, #1, line width=0.1cm] (#2.west |- #3) -- (#3);
  \end{tikzpicture}%
}
\newcommand*{\DrawArrowVH}[3][red]{%
    % #1 = draw options
    % #2 = left node
    % #3 = right node
  \begin{tikzpicture}[overlay,remember picture]
    \draw [-latex, #1, line width=0.1cm] (#2) |- (#3.east);
  \end{tikzpicture}%
}
\newcommand*{\DrawArrowHV}[3][red]{%
  \begin{tikzpicture}[overlay,remember picture]
    \draw [-latex, #1, line width=0.1cm] (#2.west) -| (#3);
  \end{tikzpicture}%
}
\newcommand*{\DrawArrowPoints}[4][red]{%
  \begin{tikzpicture}[overlay,remember picture]
    \draw [-latex, #1, line width=0.1cm] (#2) #4 to (#3);
  \end{tikzpicture}%
}


\lstnewenvironment{consolethree}[1][]
{
  \lstset{
    escapeinside={~|}{|~},
    basicstyle=\ttfamily,
    breaklines=true,
    columns=fullflexible,
    keepspaces=true,
    frame=single,
    xleftmargin=3.4pt,
    xrightmargin=3.4pt,
    belowskip=0pt,
    #1
  }
}
{}

\newcommand\includesourcethree[2][]{\lstinputlisting[frame=single,basicstyle=\ttfamily\footnotesize,
    breaklines=true,
    columns=fullflexible,
    keepspaces=true,xleftmargin=3.4pt,xrightmargin=3.4pt, #1]{#2}%
    }
