\newsection{110: Search Algorithms}
\topic{Main idea}
A pertinent idea Liow mentions is the \emph{\underline{partial solutions}} that make up a search algorithm. We describe algorithms in stages, where we perform one option such as sorting, then another like mathematical operations: but they are clearly distinct from one another. Though distinct, they help construct the piece of the puzzle the next leg of the algorithm will focus on. This is very similar to the issue of sub-problems he mentions that we deal with in recursion or general problem solving.


\newsection{110.1: Backtrack}
\topic{Decision-making}
The idea of search algorithms is to make decisions. We want the decision to be correct as much as possible, but when it isn't we are left with the last possible option: return from where we came and make another (maybe educated) decision in order to try something that may lead us to a correct solution.

This is exemplified best by \textbf{depth-first searches}, where we exhaust a chain of decisions quickly to find a proper answer. Honestly reading this back as I write it, its starting to make a lot of sense. The idea for backtracking is like \centerthis{``i went this way for a little bit, i didnt quite find what i was looking for but if i go back a few paces, maybe the turns back behind me lead to the correct solution''}

at least i hope thats what he was getting at


\newsection{110.2 -> 110.6: Pseudocode}

\newsection{110.10: Magic Square}

\newsection{110.11: Sudoku Problem}
