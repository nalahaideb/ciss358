\newsection{110: Search Algorithms}
\topic{Main idea}
A pertinent idea Liow mentions is the \emph{\underline{partial solutions}} that make up a search algorithm. We describe algorithms in stages, where we perform one option such as sorting, then another like mathematical operations: but they are clearly distinct from one another. Though distinct, they help construct the piece of the puzzle the next leg of the algorithm will focus on. This is very similar to the issue of sub-problems he mentions that we deal with in recursion or general problem solving.


\newsection{110.1: Backtrack}
\topic{Decision-making}
The idea of search algorithms is to make decisions. We want the decision to be correct as much as possible, but when it isn't we are left with the last possible option: return from where we came and make another (maybe educated) decision in order to try something that may lead us to a correct solution.

This is exemplified best by \textbf{depth-first searches}, where we exhaust a chain of decisions quickly to find a proper answer. Honestly reading this back as I write it, its starting to make a lot of sense. The idea for backtracking is like \centerthis{``i went this way for a little bit, i didnt quite find what i was looking for but if i go back a few paces, maybe the turns back behind me lead to the correct solution''}

at least i hope thats what he was getting at

\dated{1-22-26}
\newsection{110.2: Pseudocode}

So after reading some more, theres a \EMPHASIZE{very} important point i want to jot down for later:

\textlg{\textit{backtracking should distinguish between ``forward'' and ``backward''}}

What does that mean? \underline{\textbf{``forward''}} attempts to build an entire solution from the larger subset of smaller \underline{\textit{``partial solutions''}}. Sometimes its right, sometimes its wrong. if its right, then awesome we found a solution.

\EMPHASIZE{BUT}, if its \emph{wrong}, we now have to define \underline{\textbf{``backward''}}. Backward in this case is trying to go back to a point in time where we had more \emph{potentially} correct options, more potential \textbf{partial solutions} we could use to build towards making a whole solution.


\topic{Emphasis on efficiency}

The above bit of text is explaining the general, bare-bones idea of a backtracking search solution. But to make it efficient there cant be too many items that comprise the solution.

For example, if we know that a solution has options \verb!op_0, op_1, ... op_n!, and we start building a concrete solution with options \verb!op_0, op_5, op_10!, we wouldnt include them the next time we have to backtrack ALL the way back to that first iteration of finding a solution.

On top of that, (hehe foreshadowing puns) it would be more prudent to \emph{add} the current working partial solution \emph{on top of} the existing solution instead of making separate objects for each and every stage of a partial solution fitting our overall algorithm.

\newsection{110.2: Knight's tour}

my solution for the \texttt{3 $\times$ 4} knight's tour question:

\begin{tabular}{|c|c|c|c|}
\hline

\textlg{9} & \textlg{6} & \textlg{1} & \textlg{4} \\
\hline

\textlg{0} & \textlg{3} & \textlg{8} & \textlg{11} \\
\hline

\textlg{7} & \textlg{10} & \textlg{5} & \textlg{2} \\
\hline
\end{tabular}

I mean i guess its one solution, but whatever.

Anyways, i read from pg 20 to like pg 50 which is mostly a demo of some code  for finding the knights tour of a $n \times m$ board, but it showed off only 3 $\times$ 3 and 5 $\times$ 5.

I was reading it and the basic synopsis of how i should approach backtracking (for this problem at least) was like this:
\begin{itemize}

\item Have a starting point (in this case the top left-most cell in the chessboard)
\item continue building the solution from the last valid starting point, lets call it $p$
\item if all options in that ``tree'' past $p$ are failures, then go up past $p$ and choose $p + 1$ or whatever other iteration of $p$ is next
\item repeat the process until a solution is found or there is no possible solution and all other options have been exhausted

\end{itemize}

Essentially what i just said but in code form:

\begin{consolethree}
ALGORITHM: NONRECURSIVE-BACKTRACK

let s be an empty stack
push the empty solution onto s
perform_goal_test = FALSE

direction = "down"

while s is not empty:

      if perform_goal_test:
         if s is solution:
            return s
         perform_goal_test = FALSE

      let t be the top of stack s

      if direction is "down":
         if t has a child:
            push child of t onto s
            perform_goal_test = TRUE
      else:
         direction = "right"
\end{consolethree}

and this is the code i mentioned earlier:
%\begin{console}
//page 17 chapter 110


#include <iostream>
#include <iomanip>

const int ROWS = 5;
const int COLS = 5;

void pprint(int path[ROWS * COLS][2],
            int & pathlen)
{
  int x[ROWS][COLS] = {{0}};
  for (int i = 0; i < pathlen; i++)
  {
    int r = path[i][0];
    int c = path[i][1];
    x[r][c] = i + 1;
  }
  for (int i = 0; i < COLS; i++) std::cout << "+--";
  std::cout << "+\n";
  for (int r = 0; r < ROWS; r++)
  {
    for (int c = 0; c < COLS; c++)
    {
      std::cout << '|';
      if (x[r][c] == 0)
        std::cout << std::setw(2) << ' ';
      else
        std::cout << std::setw(2) << x[r][c];
    }
    std::cout << '|' << std::endl;
    for (int i = 0; i < COLS; i++) std::cout << "+--";
    std::cout << "+\n";
  }
}


bool solve(int x[ROWS][COLS],
           int path[ROWS * COLS][2],
           int & pathlen)
{
  pprint(path, pathlen);

  if (pathlen == ROWS * COLS)
  {
    return true; // SUCCESS case
  }
  else
  {
    // compute options
    if (pathlen == 0)
    {
      // Path is empty: we can start anywhere
      for (int r = 0; r < ROWS; r++)
      {
        for (int c = 0; c < COLS; c++)
        {
          x[r][c] = 1; pathlen = 1;
          path[0][0] = r; path[0][1] = c;
          bool flag = solve(x, path, pathlen);
          if (flag)
          {
            return true;
          }
          // undo
          x[r][c] = 0; pathlen = 0;
        }
      }
    }
    else
    {
      int r = path[pathlen - 1][0], c = path[pathlen - 1][1];

      // There are 8 options (some outside the board)
      int moves[8][2] = {{-2, -1},
                         {-2, +1},
                         {-1, -2},
                         {-1, +2},
                         {+1, -2},
                         {+1, +2},
                         {+2, -1},
                         {+2, +1}};
      for (int i = 0; i < 8; i++)
      {
        int newr = r + moves[i][0];
        int newc = c + moves[i][1];
        if (0 <= newr && newr < ROWS &&
            0 <= newc && newc < COLS)
        {
          if (x[newr][newc] == 0)
          {
            // CASE: (newr, newc) not occupied
            path[pathlen][0] = newr;
            path[pathlen][1] = newc;
            x[newr][newc] = pathlen + 1;
            pathlen++;
            bool flag = solve(x, path, pathlen);
            if (flag)
            {
              return true;
            }

            // CASE: (newr, newc) not occupied
            // Backtrack
            x[newr][newc] = 0;
            pathlen--;
            std::cout << "BACKTRACK!\n";
            pprint(path, pathlen);
          }
          else
          {
            // The current cell is occupied.
          }
        }
      }
    }
  }
  return false;
}
int main()
{
  int x[ROWS][COLS] = {{0}};
  int path[ROWS * COLS][2];
  int pathlen = 0;

  bool flag = solve(x, path, pathlen);
  if (flag)
  {
    std::cout << "SUCCESS!" << std::endl;
    for (int i = 0; i < ROWS * COLS; i++)
    {
      std::cout << path[i][0] << ',' << path[i][1]
                << '\n';
    }
  }
  else
  {
    std::cout << "FAILURE!\n";
  }

  return 0;
}
\end{console}



\newsection{110.10: Magic Square}

\newsection{110.11: Sudoku Problem}


