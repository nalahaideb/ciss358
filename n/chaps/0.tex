%correctness, dated 1-14-26
\newsection{Correctness}

It is \emph{\textbf{impossible}} to design an algorithm to analyze if general algorithms are going to halt or not, emphasis on \textit{general}.
\subtext{\underline{Halting Problem}}


With specific algorithms or base-cases in such algorithms it is possible to know when and how it will halt.

\topic{Correctness via induction}

\textbf{Induction}
Accepted axioms:
\begin{itemize}
\item
  Well-Ordering Principle (WOP):
  \[ if X \subseteq \mathbb{N} = {0, 1, \dots}\] is nonempty, then X has a least element $\exists m \in X$ s.t. $m \leq x$,  $\forall x \in X$ \sidenote{``LEAST'' IN THIS CASE \textbf{ DOES NOT} mean it is WITHIN THE BOUNDS. It means there is an element that is \textit{SMALLER} THAN THE SMALLEST ELEMENT IN X \EMPHASIZE{!!!}}
\item
Let $P$ be a proposition on $\mathbb{N}$, i.e $\forall$ $n \in \mathbb{N}$, then

$P(n)$ can be true or false if
\begin{enumerate}
\item $P(n_0)$ true
\item $P(n), P(n_1), P(n_2), \dots, P(n)$ is true for all $n \geq n_0$, $n \in \mathbb{N}$
\end{enumerate}
Then $P(n)$ is true $\forall n \in \mathbb{n}$, $n \geq n_0$ \subtext{Strong Induction Method}

\item
P as above:
\begin{enumerate}
\item $P(n_0)$ true
\item $P(n) \implies P(n + 1)$ is true for all $n \geq n_0$, $n \in \mathbb{N}$
\end{enumerate}
\end{itemize}

\sidenote{\EMPHASIZE{Euler's Characteristic Formula:} A relationship for \underline{planar graphs}, which are graphs capable of being laid out on a table-top with NO crossing edges}

