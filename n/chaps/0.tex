%correctness, dated 1-14-26
\newsection{Correctness}

It is \emph{\textbf{impossible}} to design an algorithm to analyze if general algorithms are going to halt or not, emphasis on \textit{general}.
\subtext{\underline{Halting Problem}}


With specific algorithms or base-cases in such algorithms it is possible to know when and how it will halt.

\topic{Correctness via induction}

\textbf{Induction}
Accepted axioms:
\begin{itemize}
\item
  Well-Ordering Principle (WOP):
  \[ if X \subseteq \mathbb{N} = {0, 1, \dots}\] is nonempty, then X has a least element $\exists m \in X$ s.t. $m \leq x$,  $\forall x \in X$ \sidenote{``LEAST'' IN THIS CASE \textbf{ DOES NOT} mean it is WITHIN THE BOUNDS. It means there is an element that is \textit{SMALLER} THAN THE SMALLEST ELEMENT IN X \EMPHASIZE{!!!}}
\item
Let $P$ be a proposition on $\mathbb{N}$, i.e $\forall$ $n \in \mathbb{N}$, then

$P(n)$ can be true or false if
\begin{enumerate}
\item $P(n_0)$ true
\item $P(n), P(n_1), P(n_2), \dots, P(n)$ is true for all $n \geq n_0$, $n \in \mathbb{N}$
\end{enumerate}
Then $P(n)$ is true $\forall n \in \mathbb{n}$, $n \geq n_0$ \subtext{Strong Induction Method}

\item
P as above:
\begin{enumerate}
\item $P(n_0)$ true
\item $P(n) \implies P(n + 1)$ is true for all $n \geq n_0$, $n \in \mathbb{N}$
\end{enumerate}
\end{itemize}

\sidenote{\EMPHASIZE{Euler's Characteristic Formula:} A relationship for \underline{planar graphs}, which are graphs capable of being laid out on a table-top with NO crossing edges}

\subtext{Dated -> 1/16/2026}

\topic{Inductive Proof Practice}

Proof of sum of numbers (by \emph{WOP}):

Assume \[1 + 2 + 3 ... + n \neq \frac{n(n+1)}{2}\]

then, $\exists n_0 \geq 0 $ s.t. \[1 + 2 ... + n_0 \neq \frac{n_0(n_0 + 1)}{2} \]

Let $\mathbb{X} = $ ${n | n \geq 0, n \in \mathbb{N}  s.t. 1 + 2 + ... + n \neq \frac{n(n + 1)}{2}}$

Thus $\mathbb{X} \neq \null$

Hence $\exists m \in \mathbb{X}$ that is the least element


\dated{1-23-26}

Proof to-do list:
%\sout{} the elements in the list that are done
\begin{itemize}
\item
Proof methods:
\begin{itemize}
\item
direct proof (assume p, \textbf{\underline{by} xyz}, therefor p)
\item
proof by contrapositive (assume \textbf{not} p)
\item
proof by contradiction (assume p, \textbf{\underline{but} xyz}, therefor p)
\end{itemize}
\item
\textbf{W}ell \textbf{O}rdering \textbf{P}rinciple, \textbf{W.O.P}

\item
Proof by Induction (both weak and strong)

\item
Combinatorial proof
\begin{itemize}
\item number of permutations = $n!$

\item number of $r$-permutations from $n$ = $n(n - 1), \dots, (n - r + 1)$ = $\frac{n!}{r!}$
\end{itemize}
\item
pigeonhole principle

\item
proof of inclusion - exclusion \textbf{\texttt{:\^(}}
\item
\item
\end{itemize}

For these proofs, if i want to progress towards a correct answer, Dr. Liow mentioned that we should \textbf{``make up a story''}.

\EMPHASIZE{IMPORTANT!!!!}

It sounds dumb but since math is already so abstracted, grounding it in a real-world use-case scenario helps to fit the components with logical conclusions so that when i take one step in a direction that is wrong, i can just say it out loud and if it doesnt make any logical sense i can take a step in a different direction to approach a better more logically sound answer.



\newsection{GRAFFS!!!!! :DDDDD}

the same formula for $v - e + f = 2$ correlates to the relationship between the number of \verb!v = vertices! and \verb!e = edges! between a type of graph called a \emph{tree}

\proof{Euler's Handshaking Lemma}

Eschewing undirected graphs \emph{\textbf{without loops or multi-edges}},

Let $G$ be a graph.

Then, $\sum{deg(v)}{\mathbb{u} \in \mathbb{V}} = 2 |E|$, where $G$ = ($V, E$).

For the graph, \EMPHASIZE{python needs to work first ehehehe}

%% \begin{python}
%% from latextool_basic import *
%% p = Plot()
%% p += Grid(-5, -5, 5, 5)

%% print(p)
%% \end{python}

We call the points where an edge touches a node an \textbf{incident point}

In the above (a node without a label connets to a node (labeled p) at a point labeled p', from there it connects to another node that has a loop, labeled p'')

The number of incident points can be counted in 2 ways:
\begin{itemize}
\item Enumerate all edges: $e_1, \dots, e_i$
Each $e_i$ has 2 incident points, and there are no others.

Therefore the number of incident points must be \textbf{2$|E|$}

\item Enumerate all nodes: $v_1, \dots, v_m$

Each $v_i$ has deg($v_i$) incident points, and there are no others.
\end{itemize}


Hence $\sum{deg(\mathbb{v}){\mathbb{v}} \in \mathbb{V}}$ = the total number of incident points

which then = 2$|E|$!!! \EMPHASIZE{yay!!!}


